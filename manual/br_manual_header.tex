
\usepackage{graphicx,color,epsfig}
\newcommand{\manualpath}{.}
\newcommand{\inputslocation}{\manualpath/inputs/}
\graphicspath{{\manualpath/diagrams/}{\manualpath/screencaps/}{\manualpath/sampler_photos/}}

	
\usepackage{listings}
\lstset{language=}
\renewcommand\lstlistingname{File}
\renewcommand\lstlistlistingname{Files}
\lstset{numbers=none, numberstyle=\tiny, stepnumber=1, numbersep=5pt,captionpos=b,frame=single,breaklines=true,basicstyle=\small, tabsize=4}




	
\usepackage{amsgen,amsmath,amstext,amsbsy,amsopn,amssymb}

\usepackage[dvipsnames]{xcolor}



\usepackage{float}

\usepackage[english]{babel}

\usepackage[left=1.25in,right=1.25in,top=1.25in,bottom=1.25in]{geometry} %
\usepackage[singlespacing]{setspace}%line spacing, set default to double

\usepackage{array}



\usepackage{makeidx}
\makeindex
	
	
\usepackage[format=plain,indention=.5cm,margin=10pt,font={small,singlespacing}]{caption}%custom captions.
\usepackage{parskip}

	
	
\usepackage{enumitem}


% Alter some LaTeX defaults for better treatment of figures:
% See p.105 of "TeX Unbound" for suggested values.
% See pp. 199-200 of Lamport's "LaTeX" book for details.
%   General parameters, for ALL pages:
\renewcommand{\topfraction}{0.9}	% max fraction of floats at top
\renewcommand{\bottomfraction}{0.8}	% max fraction of floats at bottom
%   Parameters for TEXT pages (not float pages):
\setcounter{topnumber}{2}
\setcounter{bottomnumber}{2}
\setcounter{totalnumber}{4}     % 2 may work better
\setcounter{dbltopnumber}{2}    % for 2-column pages
\renewcommand{\dbltopfraction}{0.9}	% fit big float above 2-col. text
\renewcommand{\textfraction}{0.07}	% allow minimal text w. figs
%   Parameters for FLOAT pages (not text pages):
\renewcommand{\floatpagefraction}{0.7}	% require fuller float pages
% N.B.: floatpagefraction MUST be less than topfraction !!
\renewcommand{\dblfloatpagefraction}{0.7}	% require fuller float pages

% remember to use [htp] or [htpb] for placement

 

%scaling for the screen caps used in the manual
\newcommand{\screencapsize}{0.41}

%for inputting paramotopy input files
\newcommand{\File}[3]{
\begin{center}

\begin{minipage}{0.8\linewidth}
\lstinputlisting[
	caption={#1},
	label=file:#2]{\inputslocation#3}
\end{minipage}

\end{center}
}


%for inputting bertini real input files
\newcommand{\Filenobox}[3]{

\lstinputlisting[frame=none,
	breaklines=true,
	caption={#1},
	label=#2]{\inputslocation#3}

}

\usepackage{marginnote}
\renewcommand*{\marginfont}{\color{red}\sffamily\tiny}
\newcommand{\comment}[1]{\textcolor{red}{$\star$}\marginnote{#1}}



\usepackage[utf8]{inputenc}



\usepackage[acronym,toc,xindy]{glossaries}
\makeglossaries
\newglossaryentry{cmap}
{
    name=colormap,
    description={The set of colors that are used to vivify a figure in MATLAB},
    plural=colormaps
}
 
\newglossaryentry{dependent}
{
    name=dependency,
    description={A program (or programs) that need(s) to be installed in order for a different program to run},
    plural=dependencies
}
 
\newglossaryentry{rgbt} {
  name={RGB triple},
  description={A row vector with three columns with each entry ranging from zero to one used to identify a color},
  plural=RGB triples
}

\newglossaryentry{cpath}
{
    name=CPATH,
    description={A list of directories to be automatically searched for libraries, but are searched after all libraries tagged with \-I}
}

\newglossaryentry{lib}
{
    name=LIBRARY\_PATH,
    description={The value of LIBRARY\_PATH is a colon-separated list of directories, much like PATH. When configured as a native compiler, GCC tries the directories thus specified when searching for special linker files}
}
 
\newglossaryentry{ldlib}
{
    name=LD\_LIBRARY\_PATH,
    description={LD is the name of the unix linker. This environment variable is a colon-separated set of directories where libraries should be searched for first, before the standard set of directories and is useful when debugging a new library or using a nonstandard library for special purposes.}
} 

\newglossaryentry{path}
{
    name=PATH,
    description={The PATH environment variable is used by Cygwin applications as a list of directories to search for executable files to run.}
}

\newacronym[see={[Glossary:]{rgbt}}]{RGB}{RGB}{Red Green Blue color values\glsadd{rgbt}}

\newacronym{mpi}{MPI}{Message Passing Interface}

\newacronym{gmp}{GMP}{GNU Multiple Precision Arithmetic Library}

\newacronym{gcc}{GCC}{GNU C Compiler}

\newacronym{mpfr}{MPFR}{Multiple Precision Floating-Point Reliable}

\newacronym{mpc}{MPC}{GNU Multiple Precision Complex Library}

\newacronym{stl}{STL}{stereolithography}



\usepackage{multirow, bigstrut, booktabs}
\usepackage{longtable}
\usepackage{tabu}


\usepackage{titlesec}
\newcommand{\sectionbreak}{\clearpage}

\usepackage{fancyhdr}


\fancypagestyle{front}{

    	\fancyhf{} % sets both header and footer to nothing
		\renewcommand{\headrulewidth}{0pt}

       \fancyfoot[RO, LE]{\thepage} %Page Number
       \fancyfoot[C]{} 
}


\fancypagestyle{main}{
    	\fancyhf{} % sets both header and footer to nothing
		\renewcommand{\headrulewidth}{0pt}

       \fancyfoot[RO, LE]{\thepage} %Page Number
       \fancyfoot[C]{} 

       \fancyhead[LE,RO]{\leftmark}
       \fancyhead[LO,RE]{\rightmark}
}


	
\usepackage{afterpage}

\newcommand\blankpage{%
    \null
    \thispagestyle{empty}%
    \addtocounter{page}{-1}%
    \newpage}

\usepackage[colorlinks]{hyperref}
\hypersetup{
    colorlinks=true,       % false: boxed links; true: colored links
    linkcolor=Plum,          % color of internal links (change box color with linkbordercolor)
    citecolor=blue,        % color of links to bibliography
    filecolor=magenta,      % color of file links
    urlcolor=cyan           % color of external links
}


