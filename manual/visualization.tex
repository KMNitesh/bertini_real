
\section{Visualization}


\subsection{In Matlab}
After running Bertini\_real, the output results can be visualized in Matlab.  This section assumes that the Matlab codes for Bertini\_real are already on the Matlab path.

\begin{enumerate}
\item First, open Matlab, move to the folder in which you decomposed your object, and call \texttt{gather\_br\_samples}. This parses the output from Bertini\_real into a .mat file.
\item Then, call \texttt{bertini\_real\_plotter\-}, which creates a handle class object and facilitates selection of parts of the decomposition to view. There are many options, all of which are documents and displayed via \texttt{help bertini\_real\_plotter\-} in Matlab.
\item To run bertini\_real\_plotter with a specific option, type in Matlab\\ 
	\texttt{bertini\_real\_plotter\-(`option', `option\_argument')}, where the option\_argument will vary depending on the option you decide to alter. The options are listed below.
\end{enumerate}


\subsection{Visualization options}
\noindent
\begin{longtabu} to \textwidth {
    X[1,l]	% option
    X[1,r]	% alternatives
    X[1,c]	% Modification options
    X[3,c]	% Command line appearance
    X[2,r]}	% Description/definition		
 \\ \hline
\caption{MATLAB Visualization Options}\\
\toprule
\rowfont\bfseries
\textbf{Option} & \textbf{Default} & \textbf{Alter} & \textbf{Command Line} & \textbf{Description}  \\
 \\ \hline  \\
\endfirsthead
\caption[]{\textit{Continued from previous page}}\\
 \\ \hline
\textbf{Option} & \textbf{Default} &  \textbf{Alter} & \textbf{Command Line} & \textbf{Description}  \\
 \\ \hline \\
\endhead
\bottomrule \multicolumn{5}{r}{\textit{Continued on next page}} \\
\endfoot
\bottomrule \multicolumn{5}{r}{\textit{Br15}} \\
\endlastfoot


\texttt{`autosave'} & \texttt{`on'} & \texttt{`false'}, \texttt{`0'} & \texttt{bertini\_real\_plotter (`autosave', `false')} off & Users can automatically save a figure to the working directory or not. \\  \\ \hline \\
\texttt{`\gls{cmap}'} & \texttt{`jet'} & full list \href{http://www.mathworks.com/help/matlab/ref/colormap.html}{here} & \texttt{bertini\_real\_plotter (`colormap', @summer)}  summer colormap & Users can change the \gls{cmap} by changing the handle.  \\  \\ \hline \\
\texttt{`curve'} or \texttt{`curves'} & \texttt{`true'} & \texttt{`n'}, \texttt{`no'}, \texttt{`none'}, \texttt{`false'}, \texttt{`0'} & \texttt{bertini\_real\_plotter (`curve', `false')} disables the curves option & \texttt{bertini\_real\_plotter\-} by default lets the user display the figure's raw curves.\\  \\ \hline \\
\texttt{`faces'} & \texttt{`true'} & \texttt{`n'}, \texttt{`no'}, \texttt{`none'}, \texttt{`false'}, \texttt{`0'}  & \texttt{bertini\_real\_plotter (`faces', 'none')} makes only the option to display the raw curves will be given. & By default, the figure created in MATLAB will show both the raw curves and faces. \\  \\ \hline \\
\texttt{`filename'} or \texttt{`file'} &  &  & \texttt{bertini\_real\_plotter (`filename', `Example\_File\_Name.mat')} & \texttt{bertini\_real\_plotter\-} first searches files named \texttt{BRinfo\textasteriskcentered.mat}; if more than one, uses most recent 
 \\  \\ \hline \\
\texttt{`labels'} & \texttt{`on'} & \texttt{`n'},\texttt{`no'}, \texttt{`none'}, \texttt{`false'}, \texttt{`0'}  & \texttt{bertini\_real\_plotter (`labels', `none')} off &  \texttt{bertini\_real\_plotter\-} by default lets user apply labels to the figure.\\  \\ \hline \\
\texttt{`linestyle'} & \texttt{`-'} (solid line) & line options listed \href{http://www.mathworks.com/help/matlab/ref/primitiveline-properties.html}{here} & \texttt{bertini\_real\_plotter (`linestyle', `:')} & Used to change the line style of lines in the MATLAB figure. \\  \\ \hline \\
\texttt{`monocolor'} or \texttt{`mono'} & \texttt{`off'} & \glspl{rgbt} listed \href{http://www.mathworks.com/help/matlab/ref/colorspec.html}{here} & \texttt{bertini\_real\_plotter (`mono', `r')} creates a red figure & Used to create a mono-color figure.\\  \\ \hline \\
\texttt{`proj'} &  &  &  & Unsure of what proj does \\  \\ \hline \\
\texttt{`vertices'} or \texttt{`vert'} & \texttt{`on'} & \texttt{`n'}, \texttt{`no'}, \texttt{`none'}, \texttt{`false'}, \texttt{`0'} & \texttt{bertini\_real\_plotter (`vertices', 0)} off & MATLAB can allow the user to place vertex markers and labels on the figure.  \\  \\ \hline
\end{longtabu}









\clearpage

\subsection{3D Printing}
The following description on how to 3D print the surfaces that were decomposed from Bertini\_real is taken from \textit{Bertini\_real: Numerical Decomposition of Real Algebraic Curves and Surfaces}.\cite{BrN15} 

We strive to make three-dimensional printing of surfaces decomposed using Bertini real almost
trivial. Since even the unrefined surface decomposition is a triangulation, 3D printing is as simple as:

\begin{itemize}
\item Convert data from plaintext
\item Write to stereolithography (\gls{stl}) file
\item If not compact, solidify the surface
\item Process in model repair service
\item Graphically ensure quality
\item Print
\end{itemize}

We have been successful in printing surfaces, both compact and unbounded, those which are everywhere
smooth, those which contain cusp singularity points, and even those containing singular
curves.\cite{BrN15}


