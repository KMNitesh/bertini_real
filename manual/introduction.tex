
\section{Introduction}


Welcome to Bertini\_real, software for real algebraic geometry.  This manual is intended to help the user operate this piece of numerical software, to obtain useful and high-quality results from decomposing real algebraic curves and surfaces.

Bertini\_real is compiled software, links against a parallel version of Bertini 1 compiled as a library ({\tt libbertini-parallel}, and requires Matlab and the Symbolic Computation toolbox.  It also requires several other libraries, including a few from Boost, and an installation of MPI.  All libraries should be compiled using the same compilers and dependent libraries.  



\subsection{About this manual}
The purpose of this manual is to provide a robust, orderly, and easy to understand instructions on how to use Bertini\_real. This manual has three roles: it first serves as a description of what Bertini\_real is, followed by instructions on Bertini\_real's installation on Mac, Linux, and PC operating systems, and finally as a general reference manual for Bertini\_real.
	
This manual is here to help guide a user through the installation process, as well as act as the user's manual for Bertini\_real. If there is a section that might not be entirely clear, or is confusing to a reader, please contact us (see below for contact information), and we will try to resolve the problem. Such feedback is welcome!


\subsection{Bertini\_real product description}
	Bertini\_real is an implementation of several numerical algorithms \cite{lu2007finding,besana2013cell}, to decompose the real part of a complex curve or surface in any (tractible) number of variables.
	Some of the important features of Bertini\_real include :
\begin{itemize}
\item It is a command line program for numerically decomposing the real portion of a one- or two- dimensional complex irreducible algebraic  set in any reasonable number of variables.
\item It seeks to automate the visualization and computation of algebraic curves and surface.
\end{itemize}



\subsection{Where Bertini\_real can be found}
		
	The tarball for Bertini\_real can be downloaded at \href{http://www.bertinireal.com/download.html}{Bertini\_real.com}. 
	The visualization codes for MATLAB, they can be found at \href{https://github.com/ofloveandhate/bertini_real/tree/master/matlab_codes}{GitHub}.


	\subsection{Who is developing Bertini\_real?}
	Bertini\_real is under ongoing development by the development team, which consists of Dani Brake (University of Notre Dame), Daniel Bates (Colorado State University), Jonathan Hauenstein (University of Notre Dame), Wenrui Hao (Penn State), Andrew Sommese (University of Notre Dame), Charles Wampler (General Motors. R\&D), and Pierce Cunneen (University of Notre Dame)

	This manual was written by Dani Brake, Pierce Cunneen, Chris Lembo, and Elizabeth Sudkamp.


\subsection{Contact}
\label{sec:contact}

Dani Brake: \href{mailto:danielthebrake@gmail.com}{danielthebrake@gmail.com} -- Main implementer\\
Daniel Bates: \href{mailto:bates@math.colostate.edu}{bates@math.colostate.edu} -- Advisory \\
Jonathan Hauenstein: \href{mailto:hauenstein.edu}{hauenstein.edu} -- Advisory\\
Wenrui Hao: \href{mailto:hao.50@mbi.osu.edu}{hao.50@mbi.osu.edu} -- Advisory\\
Andrew Sommese: \href{mailto:sommese@nd.edu}{sommese@nd.edu} -- Advisory\\
Charles Wampler: \href{mailto:charles.w.wampler@gm.com}{charles.w.wampler@gm.com} -- Advisory\\
Pierce Cuneen: \href{mailto:pcuneen@nd.edu}{pcuneen@nd.edu} -- Users manual \\
Elizabeth Sudkamp: \href{mailto:esudkamp@nd.edu}{esudkamp@nd.edu} -- Users manual
% FUTURE CONTRIBUTORS, please add your name to both the 'Who is developing Bertini_real' section, as well as put in your contact email.

\subsection{Acknowledgements}

The development of Bertini\_real has been supported generously by a number of sources, including 
\begin{itemize}[noitemsep]
\item the Vincent J. and Annamarie Duncan Professor of Mathematics, at the University of Notre Dame, 
\item the University of Notre Dame, 
\item the National Science Foundation grants DMS-1025564 , DMS-1115668, and DMS-1262428, 
\item the Air Force Office of Scientific Research grant FA8650-13-1-7317, 
\item Mathematical Biosciences Institute, 
\item the Sloan Research Fellowship, 
\item the Army Young Investigators Project, and 
\item the Defense Advanced Research Projects Agency Young Faculty Award.
\end{itemize}

Finally, funding for shared facilities used in this research was provided by the Division of Computer and Network Systems: an NSF grant under award number CNS-0923386.




\subsection*{Disclaimer}

Any opinions, findings, and conclusions or recommendations expressed in this material are those of the author(s) and do not necessarily reflect the views of the National Science Foundation or any other organization.
