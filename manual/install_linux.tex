
\subsection{Instructions for GNU/Linux}
\label{sec:install_linux}





\begin{enumerate}
\item Install dependencies. You are very strongly encouraged to use the package manager provided, such as {\tt apt} or {\tt yum}, to install the dependencies.  See Section~\ref{sec:deps} for a list of dependencies.
\item Build and install Bertini1, ensuring that you are building the parallel version.  This will happen automatically if you install MPI as a dependency.  Remember, you must install from source, simply downloading the pre-built executable will not get you the built libraries.
\begin{enumerate}
	\item {\tt ./configure}
	\item {\tt make -j 4}
	\item {\tt sudo make install}
\end{enumerate}
\item Build and install Bertini\_real.  
\begin{enumerate}
	\item Clone from repo, {\tt git clone https://github.com/ofloveandhate/bertini\_real}.  
	\item {\tt libtoolize}
	\item {\tt autoreconf -i} \quad if you know how to eliminate these steps by writing a bootstrap file which works on all systems, please contribute by way of a pull request.  PR's gladly acccepted to {\tt develop}.
	\item {\tt ./configure}
	\item {\tt make -j 4}
	\item {\tt sudo make install}
	\item eat something delicious, and give thanks for this peaceful day
\end{enumerate}
\item[*]
If using Matlab for symbolic engine, ensure it is installed an on the path for the command line.

In bash, {\tt touch} \url{~/.bash_profile}, then edit it, adding {\tt export PATH=/PATH/TO/MATLAB/bin:\$PATH}.%
\footnote{Note that there are several possible files into which to place this, and this particular method of using {\tt export} is shell specific.  The {\tt csh} is a little different, for example.  Too lazy to Google it?  I gotcha: \href{https://www.google.com/\#q=add+to+path+linux&*}{search on da googs here} } 

\item[*]
If using Python for symbolic engine, install your favorite version, {\tt pip}, and the necessary modules via {\tt pip}.

\end{enumerate}


If anything went wrong, please file an issue on Github.  I want this to be an easy experience.