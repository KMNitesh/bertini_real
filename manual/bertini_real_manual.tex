\documentclass[10pt]{article}




\usepackage{graphicx,color,epsfig}
\newcommand{\manualpath}{.}
\newcommand{\inputslocation}{\manualpath/inputs/}
\graphicspath{{\manualpath/diagrams/}{\manualpath/screencaps/}}

	
\usepackage{listings}
\lstset{language=}
\renewcommand\lstlistingname{File}
\renewcommand\lstlistlistingname{Files}
\lstset{numbers=none, numberstyle=\tiny, stepnumber=1, numbersep=5pt,captionpos=b,frame=single,breaklines=true,basicstyle=\small}




	
\usepackage{amsgen,amsmath,amstext,amsbsy,amsopn,amssymb}

\usepackage[dvipsnames]{xcolor}

\usepackage[colorlinks]{hyperref}
\hypersetup{
    colorlinks=true,       % false: boxed links; true: colored links
    linkcolor=Plum,          % color of internal links (change box color with linkbordercolor)
    citecolor=green,        % color of links to bibliography
    filecolor=magenta,      % color of file links
    urlcolor=cyan           % color of external links
}


\usepackage{float}

\usepackage[english]{babel}

\usepackage[left=1.25in,right=1.25in,top=1.25in,bottom=1.25in]{geometry} %
\usepackage[singlespacing]{setspace}%line spacing, set default to double

\usepackage{array}



\usepackage{makeidx}
\makeindex
	
	
\usepackage[format=plain,indention=.5cm,margin=10pt,font={small,singlespacing}]{caption}%custom captions.
\setlength{\parskip}{1em} %custom spacing between paragraphs 
	
	
	



% Alter some LaTeX defaults for better treatment of figures:
% See p.105 of "TeX Unbound" for suggested values.
% See pp. 199-200 of Lamport's "LaTeX" book for details.
%   General parameters, for ALL pages:
\renewcommand{\topfraction}{0.9}	% max fraction of floats at top
\renewcommand{\bottomfraction}{0.8}	% max fraction of floats at bottom
%   Parameters for TEXT pages (not float pages):
\setcounter{topnumber}{2}
\setcounter{bottomnumber}{2}
\setcounter{totalnumber}{4}     % 2 may work better
\setcounter{dbltopnumber}{2}    % for 2-column pages
\renewcommand{\dbltopfraction}{0.9}	% fit big float above 2-col. text
\renewcommand{\textfraction}{0.07}	% allow minimal text w. figs
%   Parameters for FLOAT pages (not text pages):
\renewcommand{\floatpagefraction}{0.7}	% require fuller float pages
% N.B.: floatpagefraction MUST be less than topfraction !!
\renewcommand{\dblfloatpagefraction}{0.7}	% require fuller float pages

% remember to use [htp] or [htpb] for placement

 

%scaling for the screen caps used in the manual
\newcommand{\screencapsize}{0.41}

%for inputting paramotopy input files
\newcommand{\File}[3]{
\begin{center}

\begin{minipage}{0.8\linewidth}
\lstinputlisting[
	caption={#1},
	label=file:#2]{\inputslocation#3}
\end{minipage}

\end{center}
}


%for inputting bertini real input files
\newcommand{\Filenobox}[3]{

\lstinputlisting[frame=none,
	breaklines=true,
	caption={#1},
	label=#2]{\inputslocation#3}

}

\usepackage{marginnote}
\renewcommand*{\marginfont}{\color{red}\sffamily\tiny}
\newcommand{\comment}[1]{\textcolor{red}{$\star$}\marginnote{#1}}



\usepackage[utf8]{inputenc}



\usepackage[acronym,toc,xindy]{glossaries}
\makeglossaries
\newglossaryentry{cmap}
{
    name=colormap,
    description={The set of colors that are used to vivify a figure in MATLAB},
    plural=colormaps
}
 
\newglossaryentry{dependent}
{
    name=dependency,
    description={A program (or programs) that need(s) to be installed in order for a different program to run},
    plural=dependencies
}
 
\newglossaryentry{rgbt} {
  name={RGB triple},
  description={A row vector with three columns with each entry ranging from zero to one used to identify a color},
  plural=RGB triples
}

\newglossaryentry{cpath}
{
    name=CPATH,
    description={A list of directories to be automatically searched for libraries, but are searched after all libraries tagged with \-I}
}

\newglossaryentry{lib}
{
    name=LIBRARY\_PATH,
    description={The value of LIBRARY\_PATH is a colon-separated list of directories, much like PATH. When configured as a native compiler, GCC tries the directories thus specified when searching for special linker files}
}
 
\newglossaryentry{ldlib}
{
    name=LD\_LIBRARY\_PATH,
    description={LD is the name of the unix linker. This environment variable is a colon-separated set of directories where libraries should be searched for first, before the standard set of directories and is useful when debugging a new library or using a nonstandard library for special purposes.}
} 

\newglossaryentry{path}
{
    name=PATH,
    description={The PATH environment variable is used by Cygwin applications as a list of directories to search for executable files to run.}
}

\newacronym[see={[Glossary:]{rgbt}}]{RGB}{RGB}{Red Green Blue color values\glsadd{rgbt}}

\newacronym{mpi}{MPI}{Message Passing Interface}

\newacronym{gmp}{GMP}{GNU Multiple Precision Arithmetic Library}

\newacronym{gcc}{GCC}{GNU C Compiler}

\newacronym{mpfr}{MPFR}{Multiple Precision Floating-Point Reliable}

\newacronym{mpc}{MPC}{GNU Multiple Precision Complex Library}

\newacronym{stl}{STL}{stereolithography}



\usepackage{multirow, bigstrut, booktabs}
\usepackage{longtable}
\usepackage{tabu}


\usepackage{titlesec}
\newcommand{\sectionbreak}{\clearpage}




\begin{document}


\pagestyle{plain} 
	\pagenumbering{roman} 
	\setcounter{page}{1}




\thispagestyle{empty}


\begin{center}


\quad % put in a space so the following vspace command succeeds
\vspace{3in}


{\LARGE Bertini\_Real}\\[\baselineskip]
User's Manual
\vskip0.5in

\comment{need a sweet picture here}

\vfill%\vskip2in
%\includegraphics[width = 0.55\linewidth]{}

\end{center}
\null
\vfill
\begin{singlespace}
Manual by\\
Pierce Cunneen \& Daniel Brake\\
University of Notre Dame \\
ACMS \hfill \today
\end{singlespace}
\newpage





	\tableofcontents
	\eject
	\pagenumbering{arabic} 
	\setcounter{page}{1}
	\eject



\section{Introduction}


Welcome to Bertini\_real, software for real algebraic geometry.  This manual is intended to help the user operate this piece of numerical software, to obtain useful and high-quality results from decomposing real algebraic curves and surfaces.

Bertini\_real is compiled software, links against a parallel version of Bertini 1 compiled as a library, and requires Matlab and the Symbolic Computation toolbox.  It also requires several other libraries, including a few from Boost, and an installation of MPI.  All libraries should be compiled using the same compilers.  

\subsection{Contact}
\label{sec:contact}

\subsection*{Acknowledgements}
\begin{itemize}
\item  This research utilized the CSU ISTeC Cray HPC System supported by NSF Grant CNS-0923386.
\item  This material is based upon work supported by the National Science Foundation under Grants No. DMS-1025564 and DMS-1115668.
\end{itemize}

\subsection{License}
\label{sec:license}

\subsection*{Disclaimer}

Any opinions, findings, and conclusions or recommendations expressed in this material are those of the author(s) and do not necessarily reflect the views of the National Science Foundation or any other organization.




\clearpage
\section{Quick Start}
\label{sec:started}

\section{Compilation and Installation}



\clearpage
\section{Using Bertini\_Real}
\label{sec:running}

\section{Troubleshooting}

\section{Visualization}

\section{3D Printing}

\clearpage

\appendix
\section{Output Formats}


\subsection{.curve}


(num\_variables total) num\_vertices num\_edges \\
num\_V0 num\_V1 num\_midpts num\_newpts \\

indices of V0  \\
indices of V1  \\
indices of midpoints \\
indices of added\_points

projection excluding the homogeneous 0 coordinate.\\

\File{Example C.curve file. }{C.curve}{C.curve}

\subsection{.edge}


\subsection{.vert}




%\ifx\standalonemode\undefined
%
%\else
%	\begin{singlespace}
%	\bibliographystyle{ieeetr}
%	\bibliography{bibliobiblioparama}
%	\end{singlespace}
%	
%	\begin{singlespace}
%	\printindex
%	\end{singlespace}
%\fi

\end{document}
